\section{Introduction}

The field of \textit{comressed sensing} (often also called compressive sensing) treats the problem of solving an
underdetermined linear system under the assumption of sparsity.
It emerged from signal processing, where such problems often occur; many signals are sparse at least in some basis,
e.g. Fourier or wavelet.
Tools developed in compressed sensing allow to take a smaller amount of linear measurements
(linear combinations of signal components) than would normally be required to restore the signal.
These tools can also be used to \textit{compress} any sparse data for efficient storage or transmission with guarantees of
successful restoration.
Another field of application is medical imaging; with the help of compressed sensing it is possible to significally
shorten the time needed to take the image.

The list of applications goes on, but our focus will be on the mathematical side of things.
During this project I worked on three main sources: the book ``A mathematical Introduction to Compressive Sensing'' by
S.~Foucart and H.~Rauhut \cite{mathintro}, the paper ``The Convex Geometry of Linear Inverse Problems'' by V.~Chandrasekaran
et al. \cite{convexgeom}, and the paper ``Living on the Edge: Phase Transitions in Convex Programs with Random Data'' by
D.~Amelunxen et al. \cite{lote}.
We will first cover some general results from the book in sections 2 and 3 and then switch to the
study of the two papers in section 4.
We will look for connections between those papers and try to reproduce some results, while filling any gaps we find.