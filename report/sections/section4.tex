\section{Details of numerical experiments}

All of the plots shown here were done in Python.
At first, I tried to write my own implementation of the Chambolle-Pock algorithm (see~\cite{chambolle}),
but it wasn't fast enough for large numbers of tests, so I switched to \texttt{cvxpy} library.
For plots in figures \ref{fig:log} and \ref{fig:lote} I generated for each pair $(s, m)$ 10 random normal matrices and
10 $s$-sparse vectors $\x$ with non-zero components being either $1$ or $-1$.
Then corresponding vectors of measurements $\y$ were computed and the obtained problem was passed into the solver.
The result of the procedure was considered to be a success if the absolute error between $\x$ and the obtained solution
was less then $10^{-4}$.
Then the frequency of successful tests was calculated and ploted as a corresponding shade of gray, with white representing 100\%
success.

For the study of transition phase, the first major question was how to compute the statistical dimension of $\norm{\cdot}_1$.
Thankfully, the paper \cite{livingontheedge} provides us with the result below, that gives us tight bounds for the statistical dimension.

\begin{proposition}
    Let $\x \in \mathbb{R}^N$ be an $s$-sparse vector.
    Then the statistical dimension of the descent cone of the $l_1$ norm satisfies the inequality
    \begin{equation}
        \psi\left( \frac{s}{N} \right) - \frac{2}{\sqrt{sN}} \leq \frac{\delta(\mathcal{D}(\norm{\cdot}_1, \x))}{N}
        \leq \psi\left( \frac{s}{N} \right),
    \end{equation}
    where $\psi:[0,1] \rightarrow [0, 1]$ is defined as
    \begin{equation} \label{eq:inf}
        \psi(\rho) \coloneq \inf_{\tau \geq 0} \left[ \rho (1+\tau^2) +
        (1-\rho) \sqrt{\frac{2}{\pi}} \bigintssss_\tau^\infty  (u-\tau)^2  e^{-u^2/2}\mathrm{d}u \right].
    \end{equation}
\end{proposition}

The infimum in \ref{eq:inf} is achieved for the unique value of $\tau$ that solves the equation
\begin{equation} \label{eq:tau_eq}
    \sqrt{\frac{2}{\pi}}\bigintssss_{\tau}^{\infty} \left( \frac{u}{\tau} - 1 \right) e^{-u^2/2} \mathrm{d} u = \frac{\rho}{1 - \rho}.
\end{equation}
Integrals in \ref{eq:inf} and \ref{eq:tau_eq} can be simplified with the use of the error function (erf) to obtain more suitable
for computation quantities:
    \begin{gather}
        \sqrt{\frac{2}{\pi}}\tau^{-1}e^{-\tau^2/2} + \op{erf}\left( \frac{\tau}{\sqrt{2}} \right) - \frac{1}{1-\rho} = 0,
        \\
        \psi(\rho) \coloneq \inf_{\tau \geq 0} \left[ \rho (1+\tau^2) +
        (1-\rho) \left( \sqrt{\frac{2}{\pi}} \tau e^{-\tau^2/2} + (1+\tau^2)\left(\mathrm{erf}\left(\frac{\tau}{\sqrt{2}}\right) - 1\right) \right) \right].
    \end{gather}

Then these results were used to plot the curves corresponding to the statistical dimension and
bounds for the number of measurements from Theorem~\ref{th:lote}.

Another moment worth expanding on is how the green curve in Fig.~\ref{fig:compare} was obtained.
In the inequality (\ref{eq:log_improved}), probability depends on $m$, so to make this comparison fare,
we would have to choose a specific threshold for the probability and then compute the corresponding $m$, which is
exactly what was done.

Let $p$ be the desired minimal probability of the successful recovery.
Then we have an equation in terms of $m$:
\[ 1 - \exp \left( - \frac{1}{2} \left[\frac{m}{\sqrt{m+1}} - \sqrt {2s \log \frac{N}{s}+\frac{5}{4}s}\right]^2 \right) = p.\]
With some simple manipulations we get the solution of this equation:
\[ m = \frac{L^2 + L\sqrt{L^2 + 4}}{2}, \]
where $L = \sqrt{\log \frac{1}{(1-p)^2}} + \sqrt {2s \log \frac{N}{s}+\frac{5}{4}s}$.
\todo[inline]{Add link to github}