\section{Studying the $l_0$-minimization}

We want to recover an  $s$-sparse vector $\mathbf{x} \in \mathbb{R}^N$ knowing a vector of $m$ measurements
$\mathbf{y} \in \mathbb{R}^m$ and a measurement matrix $\mathbf{A} \in M_{m \times N}(\mathbb{R})$ with $m < N$,
such that $\mathbf{Ax=y}$.
The system is underdetermined, so we have to look for more creative ways to solve it.
One such way is to solve the corresponding $l_0$-minimization problem.
We start by giving basic definitions as seen in \cite{mathintro}.

\begin{definition}
    The \textit{support} of a vector $\mathbf{x} \in \mathbb{R}^N$ is the set of indices of its nonzero entries:
    \begin{equation*}
        \op{supp}(\mathbf{x}) = \{ j \in [\![1,N]\!] \colon x_j \neq 0 \}
    \end{equation*}
\end{definition}

\begin{definition}
    We define $\norm{\mathbf{x}}_0$ as the cardinality of $\op{supp}(\mathbf{x})$.
    We say that the vector $\mathbf{x}$ is \textit{$s$-sparse} if $\norm{\mathbf{x}}_0 \leq s$.
\end{definition}
Note that $\norm{\cdot}_0$ is not an actual norm, nor is it a semi-norm.
Now we can formalize the problem in the following form:
\begin{equation}
    \op{minimize} \ \norm{\x}_0 \ \ \op{subject\ to} \ \ \mathbf{Ax=y}. \tag{P_0}
    \label{eq:l0}
\end{equation}
Even though it is easy to prove, there is no reason for the problem~\ref{eq:l0} to be automatically equivalent
to what we started this.
To highlight this, we write it down as a small proposition below.

\begin{proposition}
    Let $\mathbf{A} \in M_{m \times N}(\mathbb{R})$, $\x \in \mathbb{R}^N$ $s$-sparse and $\mathbf{y} \in \mathbb{R}^m$.
    The following two statements are equivalent:
    \begin{enumerate}[label=(\roman*)]
        \item The vector $\x$ is the unique solution of the compressed sensing problem, i.e, it's
        the unique $s$-sparse vector such that $\mathbf{Ax=y}$.
        \item The vector $\x$ is the unique solution of \ref{eq:l0}.
    \end{enumerate}
\end{proposition}
\begin{proof}
    $(i) \Rightarrow (ii)$ If $\x$ is the only $s$-sparse vector that satisfies $\mathbf{Ax=y}$, then there exists no such
    vector $\mathbf{z}$, that $\norm{\mathbf{z}}_0 \leq \norm{\x}_0 \leq s$, which makes $\x$ the unique minimizer of \ref{eq:l0}.

    $(ii) \Rightarrow (i)$ Immediate.
\end{proof}

One of the biggest questions in this field (and the biggest in this report) is how many measurements $m$ we need for a successful recovery of
the unknown vector.
In the following theorem we denote by $\A_S$ the matrix consisting of the columns of $\A$ indexed by $S$
and by $\x_S$ -- the vector consisting of the entries of $\x$ indexed by $S$.

\begin{theorem}
    Let $\mathbf{A} \in M_{m \times N}(\mathbb{R})$  and $\mathbf{x}, \mathbf{z} \in \mathbb{R}^N$.
    The following statements are equivalent:
    \begin{enumerate}[label=(\roman*)]
        \item \label{itm:itm1} If $\mathbf{Ax = Az}$ and both $\x$ and $\mathbf{z}$ are $s$-sparse, then $\mathbf{x = z}$.
        \item $\op{Ker} \A \cap \{ \mathbf{z} \in \mathbb{R}^N \colon \norm{\mathbf{z}}_0 \leq 2s \} = \{ \mathbf{0} \}$,
        i.e., $\mathbf{0}$ is the only $2s$-sparse vector in the $\op{Ker} \A$.
        \item For every $S \subset [\![1,N]\!]$ with $\op{card}(S) \leq 2s$, the submatrix $\A_S$ is injective as a
        map from $\mathbb{R}^{\op{card}(S)}$ to $\mathbb{R}^m$.
        \item \label{itm:itm4} Every set of $2s$ columns of $\A$ is linearly independent, i.e., $\op{rang} \A \geq 2s$.
    \end{enumerate}
\end{theorem}

\begin{proof}

    $(i) \Rightarrow (ii)&$ Let $\mathbf{z} \in \mathbb{R}^N $ be $2s$-sparse and satisfy $\A \mathbf{z} = \mathbf{0}$.
    On the other hand, we also have $\A \mathbf{0} = \mathbf{0}$, so by the hypothesis it has to be that $\mathbf{z}=\mathbf{0}$.

    $(ii) \Rightarrow (i)&$ Now let $\x$ and $\mathbf{z}$ be two $s$-sparse vectors such that $\mathbf{Ax = Az}$.
    Then $\mathbf{x-z}$ is $2s$-sparse and $\mathbf{Ax - Az} = \mathbf{A(x-z)=0}$, which implies that $\mathbf{x-z} \in \op{Ker}\A$.
    By hypothesis, we conclude that $x=z$.

    $(ii) \Rightarrow (iii)$ We recall that linear map $\mathbf{A}$ is injective if and only if $\op{Ker}\A = \{\mathbf{0}\}$.
    Let $\xx \in \mathbb{R}^{N}$ be a $2s$-sparse vector, such that $\x_S \in \op{Ker} \A_S$, where $S = \op{supp}(\x)$.
    Then $\A \x = \A_S \x_S + \A_{\overline{S}} \x_{\overline{S}} = \mathbf{0} + \A_{\overline{S}}\mathbf{0} = \mathbf{0}$,
    where $\overline{S} = [\![1,N]\!] \setminus S$.
    Then by hypothesis, $\x = 0$ and $\x_S = 0$, and thus $\A_S$ is injective.

    $(iii) \Rightarrow (ii)$  Let $\x$ be $2s$-sparse, $S = \op{supp}(\x)$ and $\A_S$ an injective map.
    Suppose that $\x \in \op{Ker}\A$.
    Then by extension, $\x_S \in \op{Ker}\A_S $ and $\x_S = \mathbf{0}$.
    Thus, $\x = 0$.

    For the last two implications we have to assume that $2s \leq m$.

    $(iii) \Rightarrow (iv)$ Let $S \subset [\![1,N]\!]$, $\op{card}(S) = 2s$.
    Then $\op{rang}(A_S)=2s - \op{dim}(\op{Ker} \A_S)) = 2s-0 = 2s$.

    $(iv) \Rightarrow (iii)$ Let $S \subset [\![1,N]\!]$, $\op{card}(S) \leq 2s$.
    Then $\op{rang}(A_S)= \op{card}(S)$ and
    $\op{dim}(\op{Ker} \A_S)) = \op{card}(S)-\op{rang}(A_S) = 0$.
    Thus, $\op{Ker} \A_S = \{ \mathbf{0}\}$.
\end{proof}


The importance of this theorem is that it gives us the necessary condition for a successful recovery of all $s$-sparse vectors $\x$
from \ref{eq:l0}: the number of measurements $m$ has to be at least $2s$.
Indeed, if it is possible to reconstruct the vector by solving $l_0$-problem, then statement (i) holds, and then according
to the theorem, $\op{rang} \A \geq 2s$.
On the other hand, the rank of a matrix cannot be greater than its smallest dimension, which in this case is $m$.
This gives us the necessary condition $m \geq 2s$.

\begin{remark}
    However, in some cases we can successfully reconstruct the vector with fewer measurements $m$.
    For example, if it is possible to construct a matrix $\A$ that depends on the specific vector $\x$ we want to recover,
    then the necessary condition becomes $m=s+1$ instead (see Theorem 2.16 in \cite{mathintro}).
    Alternatively, if we are ready to sacrifice the absolute certainty of the potential recovery, then
    we can turn to random matrices, where it's possible to get recovery guarantees in probabilistic form (we will discuss this in
    more details later).
\end{remark}
%However, this lower bound is true for the most general case, where matrix $A$ is arbitrary.
%In reality, there are many cases where the lower threshold is much softer.
%For example, in the case of random matrices, we can find a lower value of $m$ that is enough for the problem to be solved with a reasonably high probability).
%This fact will be further studied in the next sections.
%For now, we will only state a theorem from [math intro to cs], that looks at this problem from another angle: here, we assume that the vector
%$\x$ is known and that we can choose a measurement matrix by ourselves.

%\begin{theorem}
%    For any $N \geq s+1$, given an $s$-sparse vector $\x \in \mathbb{R}^N$, there exists a measurement matrix $\A \in M_{m\times N}(\mathbb{R})$
%    with $m=s+1$ rows such that the vector $\x$ can be recovered from its measurement vector $\mathbf{y=Ax}$ by solving \ref{eq:l0}.
%\end{theorem}

Unfortunately, as good as it seems in theory, the $l_0$-minimization is not effective in practice: problem \ref{eq:l0} happens to be NP-hard
(see, for example, section 2.3 in \cite{mathintro}).
The same can be said for any $l_p$-minimization problem with $p<1$, which means that we have to look for convex alternatives instead,
for example, $l_1$-minimization.